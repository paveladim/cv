%%%%%%%%%%%%%%%%%%%%%%%%%%%%%%%%%%%%%%%%%
% Medium Length Professional CV
% LaTeX Template
% Version 2.0 (8/5/13)
%
% This template has been downloaded from:
% http://www.LaTeXTemplates.com
%
% Original author:
% Trey Hunner (http://www.treyhunner.com/)
%
% Important note:
% This template requires the resume.cls file to be in the same directory as the
% .tex file. The resume.cls file provides the resume style used for structuring the
% document.
%
%%%%%%%%%%%%%%%%%%%%%%%%%%%%%%%%%%%%%%%%%

%----------------------------------------------------------------------------------------
%	PACKAGES AND OTHER DOCUMENT CONFIGURATIONS
%----------------------------------------------------------------------------------------

\documentclass{resume} % Use the custom resume.cls style

\usepackage[left=0.75in,top=0.6in,right=0.75in,bottom=0.6in]{geometry} % Document margins

\name{Павел Петров} % Your name
\address{Нижний Новгород} % Your address
\address{+7$\cdot$(930)$\cdot$666$\cdot$8600 \\ \href{mailto:paveladim@yandex.ru}{paveladim@yandex.ru} \\ \url{github.com/paveladim}} % Your phone number

\begin{document}

%----------------------------------------------------------------------------------------
%	EDUCATION SECTION
%----------------------------------------------------------------------------------------

\begin{rSection}{Образование}

{\bf ННГУ им. Н.И. Лобачевского} \hfill {\em 2018 - 2022} \\ 
Бакалавр прикладной математики и информатики \\
Кафедра теории управления и динамики систем

\end{rSection}

%----------------------------------------------------------------------------------------
%	WORK EXPERIENCE SECTION
%----------------------------------------------------------------------------------------

\begin{rSection}{Опыт работы}

Без опыта

\end{rSection}

\begin{rSection}{Стажировки}

{\bf Orion Innovation} \hfill {\em июнь 2021 - октябрь 2021} \\ 
Стажёр-программист C++

\end{rSection}

\begin{rSection}{Публикации, конференции}

{\bf XXI международная конференция $"$Математическое моделирование и суперкомпьютерные технологии$"$} \hfill {\em } \\ 
Городецкий С.Ю., Петров П.В. О диагональной реализации методов многоэкстремальной оптимизации с ограничениями для класса функций с неизмеряемыми липшицевыми производными по направлениям // Математическое моделирование и суперкомпьютерные технологии. Труды XXI Международной конференции (Н. Новгород, 22–26 ноября 2021 г.) / Под ред. проф. Д.В. Баландина. Нижний Новгород: Изд-во Нижегородского госуниверситета, 2021. – 423 с. 2021. С. 93-99.

\end{rSection}

%----------------------------------------------------------------------------------------
%	TECHNICAL STRENGTHS SECTION
%----------------------------------------------------------------------------------------

\begin{rSection}{Навыки}

\begin{tabular}{ @{} >{\bfseries}l @{\hspace{6ex}} l }
Языки программирования & C++ \\
Языки & Русский (родной), Английский (B1)
\end{tabular}

\end{rSection}

%----------------------------------------------------------------------------------------
%	EXAMPLE SECTION
%----------------------------------------------------------------------------------------

%\begin{rSection}{Section Name}

%Section content\ldots

%\end{rSection}

%----------------------------------------------------------------------------------------

\end{document}
